Denken Sie daran, ein Lerntagebuch für dieses Blatt zu führen und mit der Lösung hochzuladen!

Die Vorgaben zu diesem Blatt finden Sie im Repo {\ttfamily git@git03-\/ifm-\/min.\+ad.\+fh-\/bielefeld.\+de\+:cagix/sp-\/w21-\/vorgaben.\+git.}

Definieren Sie in der Datei {\bfseries \hyperlink{segmentanzeige_8h}{ledanzeige/segmentanzeige.\+h}} folgende Strukturen\+:


\begin{DoxyEnumerate}
\item Definieren Sie einen vorzeichenlosen Datentyp byte, der Werte von 0 bis 255 halten kann, also 8 Bit “breit” ist.
\item Definieren Sie einen Aufzählungstyp segment für die Segmente der Anzeige mit den Elementen S\+E\+G1 (Wert 0), S\+E\+G2 (Wert 1), S\+E\+G3 (Wert 2), S\+E\+G4 (Wert 3).
\item Definieren Sie einen Aufzählungstyp dot für den Dezimalpunkt der Anzeige mit den Elementen O\+FF (Wert 0) und ON (Wert 1).
\item Definieren Sie einen Aufzählungstyp brightness für die Helligkeit der Anzeige mit den Elementen D\+A\+RK (Wert 0), M\+E\+D\+I\+UM (Wert 1) und B\+R\+I\+G\+HT (Wert 7).
\end{DoxyEnumerate}

Umgang mit Basisdatentypen und Strukturen

Implementieren Sie in der Datei \hyperlink{segmentanzeige_8c}{ledanzeige/segmentanzeige.\+c} die Funktion void \hyperlink{segmentanzeige_8c_a09f1e0171b38b9abbb5d351364cfec71}{T\+M1637\+\_\+write\+\_\+byte(byte wr\+\_\+data)}, mit der wie nachfolgend beschrieben ein Byte wr\+\_\+data an die L\+E\+D-\/\+Segmentanzeige übertragen wird.

Die Datenübertragung erfolgt bitweise seriell. Sie müssen zur Übertragung eines Bytes alle 8 Bits einzeln übertragen, beginnend “von rechts”, d.\+h. mit dem niedrigstwertigen Bit (least significant bit, L\+SB)\+:


\begin{DoxyEnumerate}
\item Setzen Sie den Clock-\/\+Pin auf L\+OW\+: Aufruf von digital\+Write(\+P\+I\+N\+\_\+\+C\+L\+O\+C\+K, L\+O\+W)
\item Schreiben Sie das Bit auf den Daten-\/\+Pin\+: Aufruf von digital\+Write(\+P\+I\+N\+\_\+\+D\+A\+T\+A, L\+O\+W) (falls Sie den Wert 0 ausgeben wollen; für den Wert 1 ersetzen Sie L\+OW durch H\+I\+GH)
\item Setzen Sie den Clock-\/\+Pin auf H\+I\+GH\+: Aufruf von digital\+Write(\+P\+I\+N\+\_\+\+C\+L\+O\+C\+K, H\+I\+G\+H)
\end{DoxyEnumerate}

Nach jedem Aufruf von digital\+Write() müssen Sie mit Hilfe von delay\+Microseconds(\+D\+E\+L\+A\+Y\+\_\+\+T\+I\+M\+E\+R) kurz warten, damit sich die Spannung am Pin stabilisieren kann.

Nach dem Senden der 8 Bit wird die Datenübertragung mit dem Aufruf T\+M1637\+\_\+ack() abgeschlossen.

Die Pins des genutzten Ports sind in der Vorgabe (\hyperlink{_t_m1637__intern_8h_source}{ledanzeige/\+T\+M1637\+\_\+intern.\+h}) als Präprozessordirektiven (P\+I\+N\+\_\+\+C\+L\+O\+CK und P\+I\+N\+\_\+\+D\+A\+TA) festgelegt. Sie können diese Literale in Ihrem Code wie Konstanten verwenden.

Die genannten Symbole finden Sie in den Headern \hyperlink{_t_m1637__intern_8h_source}{ledanzeige/\+T\+M1637\+\_\+intern.\+h} und wiring\+Pi.\+h.

Umgang mit Funktionen und Header-\/\+Dateien

Schreiben Sie ein Programm zur Demonstration der L\+E\+D-\/\+Segmentanzeige-\/\+Funktionen\+: Lassen Sie beispielsweise mit zeitlichem Abstand bestimmte Zahlen anzeigen.

Nutzen Sie dazu void T\+M1637\+\_\+display\+\_\+number(float number) aus \hyperlink{_t_m1637_8h_source}{ledanzeige/\+T\+M1637.\+h}.

Rufen Sie vor der Arbeit mit der L\+E\+D-\/\+Segmentanzeige einmal die Vorgabefunktion T\+M1637\+\_\+setup() auf. Damit wird die Kommunikation mit der L\+E\+D-\/\+Segmentanzeige initialisiert.

Fügen Sie die Option -\/lwiring\+Pi (“minus klein-\/el”) zu Ihren gcc-\/\+Optionen hinzu, damit die Bibliothek beim Linken berücksichtigt wird.

Kompilieren eines Programms, Einbinden von eigenen und Standard-\/\+Headern sowie Bibliotheken

Schreiben Sie ein Makefile für Ihr Programm entsprechend den Regeln für das Praktikum. 