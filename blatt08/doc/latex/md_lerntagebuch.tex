

 title\+: \textquotesingle{}Lerntagebuch zur Bearbeitung von Blatt 02\textquotesingle{} author\+:
\begin{DoxyItemize}
\item Finn Bechinka (\href{mailto:finn.bechinka@fh-bielefeld.de}{\tt finn.\+bechinka@fh-\/bielefeld.\+de})
\item Michel-\/\+Andre Witt (\href{mailto:michel-andre.witt@fh-bielefeld.de}{\tt michel-\/andre.\+witt@fh-\/bielefeld.\+de})
\item Dennis Edler (\href{mailto:dennis.edler@fh-bielefeld.de}{\tt dennis.\+edler@fh-\/bielefeld.\+de}) ...
\end{DoxyItemize}\hypertarget{md_lerntagebuch_autotoc_md0}{}\section{Aufgabe}\label{md_lerntagebuch_autotoc_md0}
\hypertarget{md_lerntagebuch_autotoc_md1}{}\subsection{Datentypen\+:}\label{md_lerntagebuch_autotoc_md1}

\begin{DoxyEnumerate}
\item Einen Datentypen byte definieren welchen 8 Bit speichern kann.
\item Drei enmuns mit vorgegebenen Elementen und Werten definieren.
\end{DoxyEnumerate}\hypertarget{md_lerntagebuch_autotoc_md2}{}\subsection{Funktionen\+:}\label{md_lerntagebuch_autotoc_md2}
Eine Funktion {\ttfamily \hyperlink{segmentanzeige_8c_a09f1e0171b38b9abbb5d351364cfec71}{T\+M1637\+\_\+write\+\_\+byte(byte wr\+\_\+data)}} in welcher mit dem im folgendem beschriebendem Algorithmus bits (von wr\+\_\+data) an die L\+E\+D-\/\+Segmentanteige übertragen wird\+: Alle Bits werden eizeln von Rechts nach Links laufend übertragen.


\begin{DoxyEnumerate}
\item Clock-\/\+Pin auf L\+OW setzen \+: {\ttfamily digital\+Write(\+P\+I\+N\+\_\+\+C\+L\+O\+C\+K, L\+O\+W)}
\item Daten-\/\+Pin auf L\+OW (bit == 0) oder H\+I\+GH (bit == 1) setzen \+: {\ttfamily digital\+Write(\+P\+I\+N\+\_\+\+D\+A\+T\+A, \mbox{[}\+L\+O\+W O\+D\+E\+R H\+I\+G\+H\mbox{]})}
\item Clock-\/\+Pin auf H\+I\+GH setzen \+: {\ttfamily digital\+Write(\+P\+I\+N\+\_\+\+C\+L\+O\+C\+K, H\+I\+G\+H)}
\item Übertragung abschließen \+: {\ttfamily T\+M1637\+\_\+ack()}
\end{DoxyEnumerate}

Nach jedem {\ttfamily digital\+Write()} Aufruf\+: {\ttfamily delay\+Microseconds(\+D\+E\+L\+A\+Y\+\_\+\+T\+I\+M\+E\+R)} aufrufen.\hypertarget{md_lerntagebuch_autotoc_md3}{}\subsection{L\+E\+D-\/\+Demo\+:}\label{md_lerntagebuch_autotoc_md3}
Zum Demonstrieren der L\+E\+D-\/\+Segmenteinzeige-\/\+Funktionen unterschiedliche Zahlen ausgeben mit {\ttfamily T\+M1637\+\_\+display\+\_\+number(float number)}. Vor der Arbeit mit der L\+E\+D-\/\+Segmentanzeige einmal {\ttfamily T\+M1637\+\_\+setup()} aufrufen um die Kommunikation zu initialisieren.\hypertarget{md_lerntagebuch_autotoc_md4}{}\subsection{Make\+:}\label{md_lerntagebuch_autotoc_md4}
Ein makefile für das Programm erstellen.\hypertarget{md_lerntagebuch_autotoc_md5}{}\section{Ansatz und Modellierung}\label{md_lerntagebuch_autotoc_md5}
\hypertarget{md_lerntagebuch_autotoc_md6}{}\subsection{Datentypen\+:}\label{md_lerntagebuch_autotoc_md6}

\begin{DoxyEnumerate}
\item Mit Hilfe eines typedefs aus einem unsigned char byte definieren, da chars 8 Bit groß sind.
\item 3 enums mit den angegebenen Elementen und Werten erstellen.
\end{DoxyEnumerate}\hypertarget{md_lerntagebuch_autotoc_md7}{}\subsection{Funktionen\+:}\label{md_lerntagebuch_autotoc_md7}
Wie schon im Blatt01 gemacht, der Reihe nach mit einer Maske gucken ob ein Bit gesetzt ist oder nicht und dann dem entsprechent mit der {\ttfamily digital\+Write()} die Pins auf H\+I\+GH oder L\+OW setzen.\hypertarget{md_lerntagebuch_autotoc_md8}{}\subsection{L\+E\+D-\/\+Demo\+:}\label{md_lerntagebuch_autotoc_md8}
Hier einfach wie der der Aufgabe beschrieben einmal {\ttfamily T\+M1638\+\_\+setup()} aufrufen und dann mit {\ttfamily T\+M1637\+\_\+display\+\_\+number()} unterschiedliche Zahlen ausgen zwischen welchen mit {\ttfamily delay\+Microseconds} gewartet wird.\hypertarget{md_lerntagebuch_autotoc_md9}{}\subsection{Make\+:}\label{md_lerntagebuch_autotoc_md9}
Wie in der Vorlesung bzw. den {\bfseries Regeln für das Praktikum} beschrieben ein makefile für das Programm erstellen.\hypertarget{md_lerntagebuch_autotoc_md10}{}\section{Umsetzung}\label{md_lerntagebuch_autotoc_md10}
\hypertarget{md_lerntagebuch_autotoc_md11}{}\subsection{24.\+10.\+2021}\label{md_lerntagebuch_autotoc_md11}
\hypertarget{md_lerntagebuch_autotoc_md12}{}\subsubsection{Datentypen}\label{md_lerntagebuch_autotoc_md12}

\begin{DoxyItemize}
\item größtenteils Fertig 
\end{DoxyItemize}\hypertarget{md_lerntagebuch_autotoc_md13}{}\subsubsection{Funktionen}\label{md_lerntagebuch_autotoc_md13}

\begin{DoxyItemize}
\item fertig bis auf die dokumentation 
\end{DoxyItemize}\hypertarget{md_lerntagebuch_autotoc_md14}{}\subsubsection{L\+E\+D-\/\+Demo}\label{md_lerntagebuch_autotoc_md14}

\begin{DoxyItemize}
\item geht nix, keine ahung warum 
\end{DoxyItemize}\hypertarget{md_lerntagebuch_autotoc_md15}{}\subsubsection{Make}\label{md_lerntagebuch_autotoc_md15}

\begin{DoxyItemize}
\item geht auch nicht so wie wir das wollen
\end{DoxyItemize}\hypertarget{md_lerntagebuch_autotoc_md16}{}\section{Postmortem}\label{md_lerntagebuch_autotoc_md16}
tbd 